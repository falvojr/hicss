
\section{Introduction}
\label{sec:introdcution}

The adoption of information and communication technologies (ICTs) and their learning modalities have providing significant changes on the educational process. Teachers have adopting such technologies as a supporting mechanism in the classroom and to promote the continuity of studies in the learners' houses. Mobile learning (m-learning) is one of these modalities. It enables through mobile devices the learning at anyplace and everywhere. Moreover, such devices are low-cost and are widely present in the routine of young, adult and elder worldwide, creating a propitious scenario for its adoption in the educational process for several domains \cite{marcolino_edu2015, marcolinoarcht2017}.

On the other hand, the development of educational software and applications can hinder the process of adoption of ICTs' based modalities, due to issues involved, i.e., introducing pedagogical and didactic features in the solutions for supporting both teachers and learners; and guarantee the low-cost of the development of software that really met the needs of teachers and educators. An example of educational domain that need to overcome several problems is the teaching of programming \cite{souza2015}.

The majority of applications in the teaching of programming domain supports informal learning and do not provide a good level of feedback \cite{marcolino_edu2015}. Informal learning is the process of learning which there is no support of a human mediator, such as a teacher or tutor. The application itself is responsible for providing feedback, which difficult its adoption since it do not support widely teachers in classroom. The feedback gave by the applications do not show exactly the students' limitations, consequently teachers still need to monitoring in person their students in the learning process, being the main mediator between content and the learners' difficulties. 


Additionally, the reduced mobile devices keyboards and the few use of sensors and touchscreen interactions may demotivated the learners to program in such platform. In this perspective, there is a need of researches that consider the adoption of interactions seen in applications of social network and games for providing better educational applications, besides to identifying the features that really fit as an real improvement of learning for this modality \cite{marcolino_edu2015}.


In this perspective, the software product line (SPL) approach was selected for supporting the creation of mobile learning applications for the teaching of programming. The several issues and concerns involved in the educational area and in learning applications can benefit themselves from this methodology, since the teachers have the power to decide which features they expect in the applications for supporting them in classrooms, e.g., through an SPL it is possible to select features of content and learning activities, or only activities, supported by one or more programming languages. Besides, the applications can integrate different means of interaction and formats of learning content and activities \cite{marcolino_spl2015, marcolinoarcht2017}. Marcolino and Barbosa \cite{marcolinoarcht2017} presented an SPL conceptual architecture, conducting the three first sub--processes from the SPLE (software product line engineering) \cite{6799220, pohl2005}: product management, domain requirements engineering and domain design. 


The conceptual model \cite{marcolinoarcht2017} was evaluated by 31 experts from academic area (software engineers, programming teachers, m-learning experts and mobile developers), suggesting the architecture design is feasible. To complete such domain design activities, this study aims at: (i) evolving the conceptual architecture in a industry practitioners perspective through a re-analysis of the evaluated architecture, (ii) discussing the proposed improvements based on pedagogical and didactic issues; (iii) representing the improved architecture with UML (Unified Modeling Language) and SMarty (Stereotype-based Management of Variability); and (iv) presenting technique and methodology adopted for the conduction of domain realisation and domain testing SPLE sub-processes.

The research contributes towards: (i) easing SPL adoption for the development of mobile learning applications; (ii) discussing new technologies for supporting a better integration and reuse of pedagogical and didactic features; and (iii) improving the approaches and frameworks adopted considering mainly variables as developers team and time available.


This paper is organized as follows. Section \ref{sec:background} presents the main concepts about m-learning and the teaching of programming, the proposed m-learning SPL and the SMarty approach. Section \ref{sec:spl_project} presents the evolution of the SPL conceptual architecture process. Section \ref{sec:spl_smarty} presents and discusses the architecture represented with UML and SMarty. Finally, Section \ref{sec:conclusions} presents the conclusions and perspectives for future work.

