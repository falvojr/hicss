

\section{Conclusions and Future Work}
\label{sec:conclusions}

In this paper we discussed the evolution of a conceptual SPL architecture for the development of mobile learning applications for the teaching of programming based on the analysis of practitioners from the industry. The first model considered only theoretical and academic views for its evaluation. To provide a more real industrial perspective, two practitioners were integrated to our research, giving contributions to the improvement of the project.

Based on the evolved model, the component architecture diagram was presented. The model encompasses the multi--tenancy concept, microservices and repository pattern. Furthermore, the content feature was represented with UML diagram and the SMarty approach.

All the decisions and models proposed lead to a possible solution for some problems faced on the domain of teaching and programming through mobile learning applications. The discussions presented also allow the adoption of these decisions in other projects which need to deal with several educational features and short releases, besides to promote the reuse of these features.

Our next steps are the finishing of the analysis performed with teachers of programming to define our first set of features to be developed and integrate the core asset of the line. Additionally, the proposed integration of XP and TDD needs to be carefully registered to allow the identification of its positive and negative impacts on the development process.